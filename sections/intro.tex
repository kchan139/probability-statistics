\section{Introduction}

The Central Processing Unit (CPU) is often referred to as the "brain" of the computer due to its fundamental role in executing instructions and managing the operations of other components. It processes data, performs calculations, and manages tasks, making it a critical component that directly impacts a computer's performance and efficiency. As technology continues to advance rapidly, the variety and complexity of CPUs available in the market have also increased, necessitating more sophisticated methods to evaluate and predict their pricing.\\

Predicting CPU prices accurately is crucial for several reasons. For manufacturers, understanding the factors that influence CPU prices can aid in developing competitive pricing strategies, optimizing production costs, and targeting the right market segments. Accurate price predictions can help manufacturers maintain a balance between profitability and market share. For consumers, knowledge of CPU pricing dynamics enables informed purchasing decisions, ensuring that they obtain the best value for their money. This is particularly important given the diverse range of CPUs available, each with different specifications and price points.\\

This report focuses on the analysis of CPU specifications to predict prices using a variety of statistical methods. The dataset used in this study consists of detailed specifications of Intel CPUs, one of the leading CPU manufacturers in the world. Intel CPUs are widely used in various computing devices, from desktops and laptops to servers and workstations, making them an ideal subject for this study.\\

The primary goal of this report is to identify the key features that significantly influence CPU prices and to develop predictive models that can accurately estimate these prices based on the identified features. To achieve this, we employ several statistical techniques, including Analysis of Variance (ANOVA), Pearson correlation coefficient, and regression analysis. Each of these methods plays a vital role in understanding the relationships between different CPU specifications and their corresponding prices.\\

Analysis of Variance (ANOVA) is utilized to determine the impact of categorical variables on CPU prices. By comparing means across different groups, ANOVA helps identify whether certain categorical features, such as CPU series or generation, significantly affect pricing. The Pearson correlation coefficient, on the other hand, measures the strength and direction of the linear relationship between continuous variables, such as clock speed, number of cores, number of threads, and cache size, and CPU prices. This analysis helps in identifying the most influential continuous variables that should be included in the predictive models.\\

The regression analysis forms the core of our predictive modeling approach. Linear regression is initially applied to provide a baseline understanding of the linear relationships between the selected features and CPU prices. However, given the complexity of CPU pricing, we also implement polynomial regression to capture non-linear interactions among the variables. By comparing the performance metrics of these models, such as R-squared and Mean Squared Error (MSE), we can determine the most effective model for predicting CPU prices.\\

Data preprocessing is a crucial step in this analysis, ensuring the reliability and accuracy of our models. This involves handling missing values, normalizing data, and encoding categorical variables. Proper preprocessing enhances the quality of the dataset, making it suitable for rigorous statistical analysis and modeling.\\

In summary, this report aims to provide a comprehensive analysis of Intel CPU specifications to predict their prices. By leveraging statistical methods and regression models, we seek to develop robust predictive models that can assist manufacturers in pricing strategies and help consumers make informed purchasing decisions. The findings of this study will contribute to the broader understanding of CPU pricing dynamics and demonstrate the value of combining various statistical techniques to create accurate and reliable price predictions.\\

\newpage