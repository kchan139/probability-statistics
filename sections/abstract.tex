\section{Abstract}

In the rapidly advancing field of computer hardware technology, understanding and predicting the clock speed (frequency) of central processing units (CPUs) is crucial for both manufacturers and consumers. This project, \textbf{"Predicting Intel CPU Clock Speed Using Statistical Methods"}, aims to develop robust predictive models for CPU clock speed based on detailed specifications of Intel CPUs. By employing statistical techniques such as \textbf{Linear Regression} and \textbf{Analysis of Variance (ANOVA)}, this study seeks to identify the key features that significantly influence CPU clock speed.\\

The dataset utilized in this study comprises a comprehensive collection of Intel CPU specifications, including attributes such as the number of cores, number of threads, cache size, power consumption, and various architectural details. Data preprocessing steps involve handling missing values, normalizing data, and encoding categorical variables to ensure the dataset is suitable for rigorous statistical analysis.\\

To identify significant predictors of CPU clock speed, ANOVA is used to assess the impact of categorical variables, providing insights into how different CPU series and generations affect clock speeds. Linear regression is then employed to model and predict CPU clock speed based on these significant features. This method directly establishes the relationship between the dependent variable (clock speed) and the independent variables (CPU specifications).\\

The predictive modeling component of this project primarily relies on linear regression techniques. Linear regression provides a foundational understanding of the linear relationships between the predictors and CPU clock speed. The performance of the regression models is evaluated using key metrics such as R-squared, Mean Squared Error (MSE), and visual inspection of residual plots.\\

Our analysis reveals that features such as the number of cores, number of threads, cache size, and power consumption are significant determinants of CPU clock speed. The linear regression model offers valuable insights into the impact of these features on clock speed, allowing for accurate predictions based on the given specifications. While ANOVA provides supplementary information on the influence of categorical variables, it is the linear regression model that forms the core of our predictive analysis.\\

This project contributes to the broader understanding of CPU performance dynamics, providing a methodological framework that can be applied to other hardware components or similar predictive tasks. The findings have practical implications for manufacturers in optimizing CPU design and for consumers in making informed purchasing decisions. By leveraging statistical methods and regression analysis, this study offers a data-driven approach to predicting CPU clock speed, enhancing transparency and efficiency in the marketplace.\\


\newpage