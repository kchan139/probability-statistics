\section{Abstract}

In the rapidly advancing field of computer hardware technology, understanding and predicting the price of central processing units (CPUs) is crucial for both manufacturers and consumers. This project, "Predicting Intel CPU Prices Using Statistical Methods," aims to develop robust predictive models for CPU prices based on detailed specifications of Intel CPUs. By employing a variety of statistical techniques including Analysis of Variance (ANOVA), Pearson correlation coefficient, and regression analysis, this study seeks to identify the key features that significantly influence CPU prices.\\

The dataset utilized in this study comprises a comprehensive collection of Intel CPU specifications, including attributes such as clock speed, number of cores, number of threads, cache size, and power consumption. Data preprocessing steps involve handling missing values, normalizing data, and encoding categorical variables to ensure the dataset is suitable for rigorous statistical analysis.\\

To identify significant predictors of CPU prices, ANOVA is used to assess the impact of categorical variables, while the Pearson correlation coefficient measures the strength and direction of the relationships between continuous variables and CPU prices. These statistical methods help in narrowing down the most influential features that contribute to variations in CPU prices.\\

The predictive modeling component of this project employs both linear and polynomial regression techniques. Linear regression provides a foundational understanding of the linear relationships between the predictors and CPU prices. However, given the complexity of CPU pricing, polynomial regression is also applied to capture more intricate, non-linear interactions among the variables. The performance of these models is evaluated using key metrics such as R-squared, Mean Squared Error (MSE), and visual inspection of residual plots.\\

Our analysis reveals that features such as clock speed, number of cores, number of threads, and cache size are significant determinants of CPU prices. The linear regression model offers valuable initial insights, but the polynomial regression model significantly enhances prediction accuracy by accounting for non-linear relationships. The results underscore the importance of considering complex interactions among CPU specifications when predicting prices.\\

This project contributes to the broader understanding of CPU pricing dynamics, providing a methodological framework that can be applied to other hardware components or similar predictive tasks. The findings have practical implications for manufacturers in pricing strategy development and for consumers in making informed purchasing decisions. By leveraging statistical methods and regression analysis, this study offers a data-driven approach to predicting CPU prices, enhancing transparency and efficiency in the marketplace.\\


\newpage