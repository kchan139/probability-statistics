\section{Data Pre-processing}
The dataset utilized in this study is derived from a CSV file containing comprehensive specifications of Intel CPUs. This dataset serves as the foundation for our analysis, providing detailed information on various CPU features that are essential for understanding clock speed trends. In this section, we describe the dataset's structure, contents, and the relevance of its attributes to our study.

\subsection{Dataset Overview}
The CSV file comprises numerous rows, each representing an individual Intel CPU model. Each row contains multiple attributes describing the specifications and characteristics of the CPU. The dataset encompasses a diverse range of CPU models across different series, generations, and intended applications (e.g., consumer desktops, laptops, server-grade CPUs), ensuring a comprehensive coverage of CPU types and their respective attributes.

\subsection{Data Relevance and Usefulness}
The following attributes from the dataset are particularly relevant for analyzing CPU clock speeds:

\begin{itemize}
    \item \textbf{Processor\_Base\_Frequency}: This attribute represents the base clock speed of the CPU, measured in gigahertz (GHz). Higher base frequencies generally indicate faster processing capabilities and overall performance.
    \item \textbf{Max\_Turbo\_Frequency}: Many modern CPUs support Turbo Boost technology, which dynamically increases the clock speed under certain conditions. This attribute specifies the maximum Turbo Boost frequency achievable by the CPU, providing insights into its potential performance capabilities.
    \item \textbf{nb\_of\_Cores}: The number of cores in a CPU impacts its ability to handle multiple tasks simultaneously. CPUs with more cores can potentially support higher clock speeds under optimal conditions, affecting overall performance trends.
    \item \textbf{nb\_of\_Threads}: This attribute represents the number of threads supported by the CPU, which is often influenced by technologies like Intel's Hyper-Threading. A higher number of threads can contribute to better resource utilization and potentially higher effective clock speeds for certain workloads.
    \item \textbf{TDP (Thermal Design Power)}: The TDP indicates the maximum amount of heat the CPU cooling system needs to dissipate. This attribute is closely related to the CPU's power consumption and thermal characteristics, which can impact clock speed potential and performance.
    \item \textbf{Lithography}: This attribute refers to the manufacturing process node (e.g., 14nm, 10nm) used in producing the CPU. Advancements in manufacturing processes can lead to higher clock speeds and improved efficiency in newer CPU generations.
\end{itemize}

Our objective is to analyze historical trends in CPU clock speeds using these attributes. Understanding how clock speeds have evolved across different CPU generations, architectural improvements, and technological shifts is crucial for identifying patterns and making informed predictions. By focusing on these critical attributes, we aim to uncover insights into the factors driving CPU clock speed improvements and contributing to the overall technological progress in CPU development.

\subsection{Load Data}
\begin{lstlisting}[language=R]
    # Importing data 
    intel_cpu <- read.csv ("~/Downloads/archive/Intel_CPUs.csv")
\end{lstlisting}

\subsection{Explore Data}
\begin{lstlisting}[language=R]
    # The head () function is used to preview the first 
    # few rows of the data frame
    head (intel_cpu)
\end{lstlisting}

\subsection{Handle Missing Values}
\begin{lstlisting}[language=R]
    # Handling missing values by replacing them with the median
    numeric_columns <- sapply(intel_cpu, is.numeric)
    impute_median <- function(x) {
    x[is.na(x)] <- median(x, na.rm = TRUE)
    return(x)
    }
    intel_cpu[numeric_columns] <- lapply(intel_cpu[numeric_columns], impute_median)

    # Verify the changes
    summary(intel_cpu)
\end{lstlisting}


\subsection{Handle Outliers}

\subsection{Feature Scaling/Normalization}

\newpage