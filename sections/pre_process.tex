\section{Data Pre-processing}
The dataset utilized in this study is derived from a CSV file containing comprehensive specifications of Intel CPUs. This dataset serves as the foundation for our analysis, providing detailed information on various CPU features that are essential for understanding clock speed trends. In this section, we describe the dataset's structure, contents, and the relevance of its attributes to our study.

\subsection{Dataset Overview}
The CSV file comprises numerous rows, each representing an individual Intel CPU model. Each row contains multiple attributes describing the specifications and characteristics of the CPU. The dataset encompasses a diverse range of CPU models across different series, generations, and intended applications (e.g., consumer desktops, laptops, server-grade CPUs), ensuring a comprehensive coverage of CPU types and their respective attributes.

\subsection{Data Relevance and Usefulness}
The following attributes from the dataset are particularly relevant for analyzing CPU clock speeds:

\begin{itemize}
    \item \textbf{Processor\_Base\_Frequency}: This attribute represents the base clock speed of the CPU, measured in gigahertz (GHz). Higher base frequencies generally indicate faster processing capabilities and overall performance.
    \item \textbf{nb\_of\_Cores}: The number of cores in a CPU impacts its ability to handle multiple tasks simultaneously. CPUs with more cores can potentially support higher clock speeds under optimal conditions, affecting overall performance trends.
    \item \textbf{nb\_of\_Threads}: This attribute represents the number of threads supported by the CPU, which is often influenced by technologies like Intel's Hyper-Threading. A higher number of threads can contribute to better resource utilization and potentially higher effective clock speeds for certain workloads.
    \item \textbf{TDP (Thermal Design Power)}: The TDP indicates the maximum amount of heat the CPU cooling system needs to dissipate. This attribute is closely related to the CPU's power consumption and thermal characteristics, which can impact clock speed potential and performance.
    \item \textbf{Lithography}: This attribute refers to the manufacturing process node (e.g., 14nm, 10nm) used in producing the CPU. Advancements in manufacturing processes can lead to higher clock speeds and improved efficiency in newer CPU generations.
\end{itemize}

Our objective is to analyze historical trends in CPU clock speeds using these attributes. Understanding how clock speeds have evolved across different CPU generations, architectural improvements, and technological shifts is crucial for identifying patterns and making informed predictions. By focusing on these critical attributes, we aim to uncover insights into the factors driving CPU clock speed improvements and contributing to the overall technological progress in CPU development.

\subsection{Load Data}
In this section, we load and clean the data for our analysis.

\begin{lstlisting}[language=R]
    # Load necessary library
    library (dplyr)

    # Importing data
    intel_cpu <- read.csv ("~/Downloads/Intel_CPUs.csv")

    # Specify the desired columns
    desired_cols <- c ("Processor_Base_Frequency", "nb_of_Cores", "nb_of_Threads", "TDP", "Lithography")

    # Subset the data to only include the desired columns
    intel_cpu_subset <- intel_cpu %>% select (one_of(desired_cols))

    # Remove rows with base frequency measured in MHz
    intel_cpu_subset <- intel_cpu_subset %>%
        filter (!grepl ("MHz", Processor_Base_Frequency))

    # Convert necessary columns to appropriate types if they are not
    # Removing 'GHz' and converting Processor_Base_Frequency to numeric
    intel_cpu_subset$Processor_Base_Frequency <- 
        as.numeric (gsub (" GHz", "", intel_cpu_subset$Processor_Base_Frequency))

    # Additional conversion to ensure nb_of_Cores, nb_of_Threads, TDP, and Lithography are numeric where applicable
    intel_cpu_subset$nb_of_Cores <- 
        as.numeric (intel_cpu_subset$nb_of_Cores)
    intel_cpu_subset$nb_of_Threads <- 
        as.numeric (intel_cpu_subset$nb_of_Threads)
    intel_cpu_subset$TDP <- 
        as.numeric (gsub (" W", "", intel_cpu_subset$TDP))
    intel_cpu_subset$Lithography <- 
        as.numeric (gsub (" nm", "", intel_cpu_subset$Lithography))

\end{lstlisting}

\vspace*{1cm}

\subsection{Handle Missing Values}

In this analysis, we focus on specific attributes of Intel CPUs that directly impact their clock speeds: Processor\_Base\_Frequency, nb\_of\_Cores, nb\_of\_Threads, TDP, and Lithography. It is essential to ensure the integrity and accuracy of these data points for meaningful analysis and reliable model predictions. \textbf{Here's why using the median for imputation is not suitable for these attributes and why removing rows with missing values is a better approach}:

\begin{itemize}
    \item Processor\_Base\_Frequency:
    \begin{itemize}
        \item Contextual Integrity: The base frequency is a critical performance metric measured in GHz. Imputing a median value may introduce inaccuracies because the actual base frequencies are often specific to CPU models and designs. These values are precise and need to reflect the true performance characteristics of the CPU.
        \item Range and Units: Base frequencies vary within specific ranges, and a median value might not accurately represent the operational characteristics of the CPUs, especially when considering variations in architecture and generation.
    \end{itemize}

    \item nb\_of\_Cores and nb\_of\_Threads:
    \begin{itemize}
        \item Specific Configuration: The number of cores and threads is an intrinsic property of a CPU model, determined by its design and intended use. These are exact numbers that are integral to the CPU's capability. Imputing a median would mean assigning an arbitrary number of cores or threads that do not exist in the actual hardware configurations, leading to misleading results.
        \item Consistency: Each CPU model has a specific number of cores and threads. Removing rows with missing values ensures we only analyze data with complete and accurate configurations.
    \end{itemize}

    \item TDP (Thermal Design Power):
    \begin{itemize}
        \item Thermal Characteristics: TDP is a precise measure of the maximum heat a CPU can generate under typical load, directly influencing cooling solutions and performance. Using a median TDP might not accurately reflect the thermal design considerations specific to different CPU models.
        \item Power Management: Different CPUs have different power and thermal characteristics. It's crucial to work with exact TDP values to maintain the accuracy of the analysis.
    \end{itemize}

    \item Lithography:
    \begin{itemize}
        \item Manufacturing Process: Lithography measures the process technology node (e.g., 14nm, 10nm). These are fixed and precise values associated with the manufacturing process of the CPU. A median value would not accurately represent any actual process node and could distort the analysis of technological trends.
    \end{itemize}
\end{itemize}

\begin{lstlisting}[language=R]
    # Remove rows with missing values in the specific columns
    cols_to_check <- c ("Processor_Base_Frequency", "nb_of_Cores", "nb_of_Threads", "TDP", "Lithography")
    intel_cpu_subset <- intel_cpu_subset[complete.cases (intel_cpu_subset[, cols_to_check]), ]

    # Display the resulting subset
    print (intel_cpu_subset)
\end{lstlisting}

\subsection{Handle Outliers}
Handling outliers is essential for maintaining the integrity and accuracy of our data analysis. By identifying and removing outliers, we can ensure that our statistical analyses and predictive models are not unduly influenced by anomalous data points. This leads to more reliable insights and better overall model performance.

\begin{lstlisting}[language=R]
    intel_cpu_temp <- intel_cpu_subset
    intel_cpu_temp$TDP <- log10(intel_cpu_temp$TDP)
\end{lstlisting}

\newpage